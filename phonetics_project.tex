\documentclass[12pt]{article}
\usepackage[margin=1in]{geometry}
\usepackage{tipa}
\usepackage{phonetic}
\usepackage{pifont}
\usepackage{arydshln}
\usepackage{rotating}
\usepackage{array}
\usepackage[table]{xcolor}
\usepackage[applemac, utf8]{inputenc}
\usepackage{tikz}
\usetikzlibrary{arrows,automata}
\usepackage{enumerate}
\usepackage{textgreek}
\newcommand{\hand}{\ding{43}}
\usepackage{gb4e}
\usepackage{qtree}
\usepackage{ragged2e}
\usepackage{multicol}
\usepackage{soul}
\usepackage{tikz-qtree-compat}
\tikzset{every tree node/.style={align=center, anchor=north}}
\usetikzlibrary{positioning}
\usetikzlibrary{arrows,automata}
\usepackage{natbib}
\bibliographystyle{apa}
\title{Phonetics Final Paper}
\author{Lisa Hofmann and Andrew Hedding}
\begin{document}
\maketitle
\section{Introduction}
\begin{itemize}
	\item Sande and Hedding hypothesize that there is phonological evidence that geminates contribute weight to syllables in Amharic and singleton codas do not.
	\item They argue that syllables closed by a geminate are heavy and that the Weight-to-Stress principle is high ranked in the language, causing stress to shift to syllables with geminates. 
	\item They mostly judge stress impressionistically, with minimal phonetic evidence. 
	\item The aim of this paper is to test their hypothesis by using phonetic analysis to determine if their predictions are borne out.
	\item We will also test another claim that they make: reduplication of geminates results in a singleton copy.	
\end{itemize}
\section{Methodology}
\section{Results}
\subsection{Geminates}

\begin{exe}
\ex{Consonant Duration of Male Speaker (rounded to 4 digits)} \label{gemmale}
\begin{center} \renewcommand*\arraystretch{1.2}
\scalebox{1}[1]{\begin{tabular}[t]{|rrl|c|c|} \hline
\multicolumn{3}{|c|}{\textbf{Word}} & \textbf{Singleton Duration (in sec.)} & \textbf{Geminate Duration (in sec.)} \\[0.5ex]
\hline  \textipa{a\texttoptiebar{\textteshlig}a\texttoptiebar{\textteshlig}\texttoptiebar{\textteshlig}\textbari r} & & & 0.074 & 0.177  \\
\hline  \textipa{adaddis} & & & 0.033 & 0.149  \\
\hline  \textipa{d\textepsilon mammak'} & & & 0.052 & 0.123 \\
\hline 	\textipa{ka\texttoptiebar{\textteshlig}a\texttoptiebar{\textteshlig}\texttoptiebar{\textteshlig}\textsyllabic{n}} & & & 0.041 & 0.169 \\
\hline  \textipa{r\textepsilon\texttoptiebar{\textdyoghlig}a\texttoptiebar{\textdyoghlig}\texttoptiebar{\textdyoghlig}\textbari m} & & & 0.064 & 0.175 \\
\hline  \textipa{safaffi} & & & 0.077 & 0.158 \\
\hline  \textipa{talallak'} & & & 0.039 & 0.108 \\
\hline  \textipa{tananna\textesh} & & & 0.039 & 0.097 \\
\hline  \textbf{Mean Duration:} & & & \textbf{0.052} & \textbf{0.145} \\
\hline \end{tabular}} \renewcommand*\arraystretch{1} \end{center}
\end{exe}

\begin{exe}
\ex{Consonant Duration of Female Speaker (rounded to 4 digits)} \label{gemfemale}
\begin{center} \renewcommand*\arraystretch{1.2}
\scalebox{1}[1]{\begin{tabular}[t]{|rrl|c|c|} \hline
\multicolumn{3}{|c|}{Word} & \textbf{Singleton Duration (in sec.)} & \textbf{Geminate Duration (in sec.)} \\[0.5ex]
\hline  \textipa{a\texttoptiebar{\textteshlig}a\texttoptiebar{\textteshlig}\texttoptiebar{\textteshlig}\textbari r} & & & 0.045 & 0.185  \\
\hline  \textipa{d\textepsilon mammak'} & & & 0.113 & 0.169  \\
\hline  \textipa{hajajjal} & & & 0.115 & 0.151 \\
\hline  \textipa{r\textepsilon\texttoptiebar{\textdyoghlig}a\texttoptiebar{\textdyoghlig}\texttoptiebar{\textdyoghlig}\textbari m} & & & 0.093 & 0.189 \\
\hline  \textipa{talallak'} & & & 0.057 & 0.131 \\
\hline  \textipa{tananna\textesh} & & & 0.048 & 0.159 \\
\hline  \textipa{wufaffram} & & & 0.082 & 0.111 \\
\hline  \textbf{Mean Duration:} & & & \textbf{0.079} & \textbf{0.156} \\
\hline \end{tabular}} \renewcommand*\arraystretch{1} \end{center}
\end{exe}

\begin{itemize}
\item Can we prove that geminates are distinct from singletons?
\item Each token is predicted to have a geminate and an identical singleton. Measure the duration of each to demonstrate that geminates are longer. 	
\end{itemize}

\subsection{Stress}

\section{Discussion}

\section{Conclusion}

%\bibliography{phonetics_bib}
\end{document}
